\chapter{Conclusion}
\label{sec:conclusion}
In this master’s thesis, the problem of optimizing the morphology of several omni-directional
rotary-wing micro aerial platforms was investigated. A general MAV model was first
studied for the purpose of characterizing the drones' morphology. Different optimization
problems were developed based on different criteria defining the omni-directionality.
A tool was elaborated in the MATLAB$^\textrm{\textregistered}$ computing environment
to solve these different optimization problems. A number of optimal multi-rotor MAVs were
acquired by this tool and analyzed, thus allowing the formulation of a general "law" on the
optimal omni-directional MAV morphology for the drones concerned in this work.
While testing the different vehicles' design
in a simulation environment proved their feasibility and their controllability, the lack of
experimentation on a real prototype for the verification of different effects is still outstanding.\\
Future work therefore includes, the development of prototypes to do tests in real life to
characterize the disturbance rejection abilities of the designs, and to characterize the airflow
interactions between the propellers. These are among effects not yet accounted for in the present approach.
Furthermore, such prototypes would allow for tests of the interaction between the real world and
the presente optimal designs.
